\documentclass[12pt]{article}
\usepackage{amsmath} 
\usepackage[round]{natbib}
\usepackage{geometry}
% \usepackage{times}
\usepackage{graphicx}
% \usepackage{courier}
\usepackage{mathpazo}
\usepackage{mathrsfs}
\usepackage{bm}
\usepackage[colorlinks,linkcolor=red,anchorcolor=blue,citecolor=blue]{hyperref}
\usepackage{amsthm}
\usepackage{ulem}
\geometry{verbose,letterpaper,tmargin=1in,bmargin=.75in,lmargin=.75in,rmargin=1in}

\title{Bayesian Quantile Regression with Monotone Dropout}
\date{\today}
\author{}

\newtheorem{thm}{Theorem}[section]
\newtheorem{deff}[thm]{Definition}
\newtheorem{rmk}[thm]{Remark}
% \newtheorem{prf}[thm]{Proof}
\newtheorem{cor}[thm]{Corollary}
\newtheorem{emp}[thm]{Example}
\newtheorem{lem}[thm]{Lemma}
\newtheorem{pps}[thm]{Proposition}

\newcommand{\polya}{P\'{o}lya}
\newcommand{\iid}{\stackrel{\text{i.i.d}}{\sim}}
\DeclareMathOperator{\pr}{p}
\DeclareMathOperator{\pt}{PT}

\begin{document}

\maketitle{}

\section{Mixture of bivariate normals with missingness at $Y_2$}

Let $\bm{Y} = (Y_{1}, Y_{2})^{T}$ and $R = I\{Y_2 \text{ observed}\}$. Suppose we
are interested in the $\tau$-th marginalized quantile regression
coefficients $\bm \gamma = (\bm \gamma^{\tau}_1, \bm \gamma^{\tau}_2)$, where
\begin{align*}
  \bm \gamma_1^{\tau} & = (\gamma_{01}^{\tau}, \gamma^{\tau}_{11}), \\
  \bm \gamma_{2}^{\tau} & = (\gamma_{02}^{\tau}, \gamma^{\tau}_{12}).
\end{align*}
Such that
\begin{align}
  \label{eq:quan1}
  \pr (Y_{i1} \leq \gamma^{\tau}_{01} + x_i \gamma^{\tau}_{11} | x_i ) & = \tau ,\\
  \label{eq:quan2}
  \pr (Y_{i2} \leq \gamma^{\tau}_{02} + x_i \gamma^{\tau}_{12} | x_i ) & = \tau .
\end{align}
Here we assume covariate $x_i$ are constant across treatments or time
points, while the quantile regression coefficients $\bm \gamma^{\tau}$
are time varying.
 
Assume $\bm Y$ follows bivariate normal distribution:
\begin{align*}
  \bm{Y}|R = r &\sim N(\bm{\mu}^{(r)}, \Sigma^{(r)}), r = 0, 1,\\
  R & \sim Ber(\phi).
\end{align*}
Reparametrize the above distribution as
\begin{align}
  \label{eq:model1}
  Y_{i1}|R = r, x_i &\sim N(\Delta_{i1} + \beta_{01}^{(r)} + x_i\beta_{11}^{(r)}, \sigma_{11}^{(r)}),\\
  \label{eq:model2}
  Y_{i2}|R = r, x_i, y_{i1} & \sim N(\Delta_{i2} + \beta_{02}^{(r)} +
  x_i\beta_{12}^{(r)} + y_{i1}\beta_{22}^{(r)}, \sigma_{2|1}^{(r)}).
\end{align}
where the quantity $\Delta_{it}, t = 1, 2$ is determined by other
parameters in the model and can be solved from
\begin{align}
  \label{eq:const1}
  \tau & =  \sum_{r = 0}^1 \pr (Y_{i1} \leq \gamma^{\tau}_{01} + x_i \gamma^{\tau}_{11} | x_i , R = r) \pr (R = r), \\
  \label{eq:const2}
  \tau & = \sum_{r = 0}^{1} \pr (Y_{i2} \leq \gamma^{\tau}_{02} + x_i \gamma^{\tau}_{12} | x_i , R =
  r) \pr (R = r).
\end{align}
Specific derivation can be found in remark \ref{sec:delta}. So far we
consider $\pr (R = 1|x_i) = \pr (R = 1) = \pi = 1 - \pr (R = 0)$,
which means missingness does not depend on covariates (could be a
further topic in future research).

Meanwhile, for identifiability issue in equation (\ref{eq:model1}) and
(\ref{eq:model2}) , we put constraints on the $\beta$ parameters:
\begin{align}
\label{eq:beta01}
  \beta_{01}^{(1)} + \beta_{01}^{(0)} & = 0 ,\\
\label{eq:beta11}
  \beta_{11}^{(1)} + \beta_{11}^{(0)} & = 0, \\
  \label{eq:beta02}
  \beta_{02}^{(1)} + \beta_{02}^{(0)} & = 0, \\
  \label{eq:beta12}
  \beta_{12}^{(1)} + \beta_{12}^{(0)} & = 0, \\
  \label{eq:beta22}
  \beta_{22}^{(1)} + \beta_{22}^{(0)} & = 0.
\end{align}

\begin{rmk}[$\Delta$ Expression]\label{sec:delta}
  Expand equation (\ref{eq:const1}) with equation (\ref{eq:quan1}) and
  (\ref{eq:model1}), we have
  \begin{align}
    \tau & = \pr (R = 1 | x_i) \pr (y_{i1} \leq \gamma^{\tau}_{01} + x_i \gamma^{\tau}_{11} | x_i , R = 1) + \pr(R = 0 | x_i) \pr (Y_{i1} \leq \gamma^{\tau}_{01} + x_i \gamma^{\tau}_{11} | x_i , R = 0)  \\
\label{eq:delta10}
    & = \pi F_{i1}^{(1)} (\gamma^{\tau}_{01} + x_i \gamma^{\tau}_{11}; \Delta_{i1},
    \beta_{01}^{(1)}, \beta_{11}^{(1)}, x_i, \sigma_{11}^{(1)}) +
    (1-\pi)F_{i1}^{(0)} (\gamma^{\tau}_{01} + x_i \gamma^{\tau}_{11};
    \Delta_{i1}, \beta_{01}^{(0)}, \beta_{11}^{(0)}, x_i, \sigma_{11}^{(0)}).
  \end{align}
  Thus,
  \begin{equation}
    \label{eq:delta1}
    \Delta_{i1} = h_{i1}(\gamma^{\tau}_{01} , \gamma^{\tau}_{11}, x_{i}, \beta_{01}^{(r)}, \beta_{11}^{(r)}, x_i, \sigma_{11}^{(r)}, \pi, \tau ), r = 0, 1.
  \end{equation}

  Similarly, expand equation (\ref{eq:const2}) with equation
  (\ref{eq:quan2}) and (\ref{eq:model2}) to get
  \begin{align}
    \tau & = \pr (R = 1) \pr (Y_{i2} \leq \gamma^{\tau}_{02} + x_i
    \gamma^{\tau}_{12} | x_i , R = 1) + \pr (R = 0) \pr (Y_{i2} \leq
    \gamma^{\tau}_{02} + x_i
    \gamma^{\tau}_{12} | x_i , R = 0)  \\
    & = \pi \int \pr (y_{i2} \leq \gamma^{\tau}_{02} + x_i \gamma^{\tau}_{12} | x_i , y_{i1}, R = 1) dF_{i1}^{(1)}(y_{i1}; \Delta_{i1}, x_i, \beta_{*1}^{(1)}, \sigma_{11}^{(1)})  \\
    & \quad + (1-\pi) \int \pr (y_{i2} \leq \gamma^{\tau}_{02} + x_i \gamma^{\tau}_{12} | x_i , y_{i1}, R = 0) dF_{i1}^{(0)}(y_{i1}; \Delta_{i1}, x_i, \beta_{*1}^{(0)}, \sigma_{11}^{(0)})  \\
\label{eq:delta20}
    & = \pi \int F_{i2|1}^{(1)} (\gamma^{\tau}_{02} + x_i \gamma^{\tau}_{12} ; \Delta_{i2}, \beta_{*2}^{(1)}, \sigma_{2|1}^{(1)}, x_i , y_{i1}) dF_{i1}^{(1)}(y_{i1}; \Delta_{i1}, x_i, \beta_{*1}^{(1)}, \sigma_{11}^{(1)})  \\
    & \quad + (1-\pi) \int F_{i2|1}^{(0)} (\gamma^{\tau}_{02} + x_i \gamma^{\tau}_{12} ; \Delta_{i2},\beta_{*2}^{(0)}, \sigma_{2|1}^{(0)},  x_i , y_{i1}) dF_{i1}^{(0)}(y_{i1}; \Delta_{i1}, x_i, \beta_{*1}^{(0)}, \sigma_{11}^{(0)}).  \\
  \end{align}
  Hence,
  \begin{equation}
    \label{eq:delta2}
    \Delta_{i2} = h_{i2}(\gamma^{\tau}_{02} , \gamma^{\tau}_{12}, x_{i},y_{i1},  \beta_{*1}^{(r)}, \beta_{*2}^{(r)},  \sigma_{11}^{(r)}, \sigma_{2|1}^{(r)}, \pi, \tau, \Delta_{i1}), r = 0, 1.
  \end{equation}
\end{rmk}

\subsection{Sensitivity Analysis}
For identifiability issue in equation (\ref{eq:model2}), we put
constraints (\ref{eq:beta02} - \ref{eq:beta22}), thus we denote
\begin{align}
\label{eq:c0}
  \beta_{02}^{(1)} & = - \beta_{02}^{(0)} ,\\
\label{eq:c1}
  \beta_{12}^{(1)} & = -\beta_{12}^{(0)}, \\
\label{eq:c2}
  \beta_{22}^{(2)} & = -\beta_{22}^{(0)}, \\
\label{eq:cs}
  \sigma_{2|1}^{(0)} & = \lambda\sigma_{2|1}^{(1)}.
\end{align}
And when $R = 0$, $Y_{i2}|R = 0$ is not observed, $\bm \xi_s =
(\beta_{02}^{(0)}, \beta_{12}^{(0)}, \beta_{22}^{(0)}, \lambda)$ could
be a group of sensitivity parameters. When $\bm \xi_s = (0, 0, 0, 1)$
yields $p(Y_{i2}|y_{i1}, x_i, R = 1) = p(Y_{i2}| y_{i1}, x_i, R = 0)$,
where MAR condition satisfies.

\subsubsection{Constructing Priors}
We put priors on $(\bm \xi_s, \bm \xi_m)$ as :
\begin{displaymath}
  p(\bm \xi_s, \bm \xi_m) = p(\bm \xi_s) p(\bm \xi_m).
\end{displaymath}
If we assume MAR with no uncertainty, the prior of $\bm \xi_s$ is
$\pr(\bm \xi_s = (0, 0, 0, 1)) = 1$. Sensitivity analysis can be
executed through putting a set of priors on $\bm \xi_s$ to check the
effect of priors on the posterior inference about quantile regression
coefficients $\bm \gamma_{ij}^{\tau}$. For example, if MAR is assumed
with uncertainty, priors can be assigned as $E(\bm \xi_s) = (0, 0, 0,
1)$ with $Var(\bm \xi_s) \neq \bm 0$. If we assume MNAR with no
uncertainty, we can put priors satisfying $E(\bm \xi_s) =
\Delta_{\xi}$, where $\Delta_{\xi} \neq (0, 0, 0, 1)$ and $Var(\bm
\xi_s) = \bm 0$. If MNAR is assumed with uncertainty, then priors
could be $E(\bm \xi_s) = \Delta_{\xi}$, where $\Delta_{\xi} \neq (0,
0, 0, 1)$ and $Var(\bm \xi_s) \neq \bm 0$.

\subsection{Computation}
\subsubsection{Likelihood}
Denote the parameters $\bm \xi = (\bm \xi_m, \bm \xi_s)$, where $\bm
\xi_m = ( \gamma_{*i}^{\tau}, \beta_{*1}^{(1)}, \sigma_{11}^{(r)},
\sigma_{2|1}^{(1)}, \phi) , t= 1, 2 , r = 0, 1$ are the identifiable
parameters and $\bm \xi_s = (\beta_{02}^{(0)}, \beta_{12}^{(0)},
\beta_{22}^{(0)}, \lambda)$ are sensitivity parameters.

The likelihood for observation $\bm Y_i = (Y_{i1}, Y_{i2})$ of subject
$i$ with missing indicator $R$ is
\begin{align}
  \label{eq:like1}
  L_i(Y_{i1}, Y_{i2}, R = 1 | \bm \xi , x_i) & = \pr (R = 1 | \phi) \pr (Y_{i1} | R = 1,  \gamma_{*1}^{\tau}, \beta_{*1}^{(1)}, \sigma_{11}^{(1)}, x_i) \pr (Y_{i2} | R = 1, Y_{i1}, \gamma_{*2}^{\tau}, x_i, \beta_{*2}^{(0)}, \sigma_{2|1}^{(1)}) \\
  \label{eq:like0}
  L_i(Y_{i1}, R = 0 | \bm \xi, x_i) & = \pr (R = 0 | \phi) \pr (Y_{i1}
  | R = 0, x_i, \gamma_{*1}^{\tau}, \beta_{*1}^{(1)},
  \sigma_{11}^{(0)})
\end{align}

We are going to put priors on the parameters $\bm \xi$ and make exact
inference through posterior samples. First we illustrate how to
calculate $\Delta_{i1}, \Delta_{i2}$ given all the other parameters
$\bm \xi = (\bm \xi_m, \xi_s)$ = ($\gamma_{01}, \gamma_{11},
\gamma_{02}, \gamma_{12}, R, x_i, \beta_{01}^{(1)}, \beta_{11}^{(1)},
\beta_{02}^{(0)}, \beta_{12}^{(0)}, \beta_{22}^{(0)},
\sigma_{11}^{(1)}, \sigma_{11}^{(0)}, \sigma_{2|1}^{(1)}, \lambda,
\phi$). Omitting $\tau$ to save typing.

\begin{enumerate}
\item [$\Delta_{i1}$: ] From remark \ref{sec:delta} and expand
  equation \eqref{eq:delta10}, we have
  \begin{align}\label{eq:solve1}
    \tau & = \pi \Phi \left( \frac{\gamma_{01}+ x_i\gamma_{11} -
        \Delta_{i1} - \beta_{01}^{(1)} -
        x_i\beta_{11}^{(1)}}{\sigma_{11}^{(1)}} \right) + (1-\pi) \Phi
    \left( \frac{\gamma_{01} + x_i\gamma_{11} - \Delta_{i1} +
        \beta_{01}^{(1)} + x_i\beta_{11}^{(1)}}{\sigma_{11}^{(0)}}
    \right).
  \end{align}
  where $\Phi$ is the cdf of standard normal distribution and we use
  the constraints in equation \eqref{eq:beta01} and \eqref{eq:beta11}.
  $\Delta_{i1}$ can be solved through equation \eqref{eq:solve1} by
  all means:
  \begin{itemize}
  \item R: \textit{uniroot} function or other root function can help
    to find the root, which probably would not be used since
    ultimately the code would be realized through faster language like
    fortran or C/C++. R is too slow to evaluate those likelihood in
    this intensive computation job.
  \item Fortran: looking for method to find the root or implement it
    through Newton-Raphson method by our own.
  \end{itemize}
\item [$\Delta_{i2}$: ] Similarly follow remark \ref{sec:delta} and
  expand equation \eqref{eq:delta20}:
  \begin{align}
    \tau & = \pi \int \Phi \left( \frac{\gamma_{02} + x_i\gamma_{12} - \Delta_{i2} + \beta_{02}^{(0)} + x_i\beta_{12}^{(0)} + y_{i1}\beta_{22}^{(0)}}{\sigma_{2|1}^{(1)}} \right) dF_{i1}^{(1)}(y_{i1}; \Delta_{i1}, x_i, \beta_{*1}^{(1)}, \sigma_{11}^{(1)}) \\
    \label{eq:solve2}
    & \quad + (1-\pi) \int \Phi \left( \frac{\gamma_{02} + x_i\gamma_{12} - \Delta_{i2} - \beta_{02}^{(0)} - x_i\beta_{12}^{(0)} - y_{i1}\beta_{22}^{(0)}}{\lambda\sigma_{2|1}^{(1)}} \right) dF_{i1}^{(0)}(y_{i1}; \Delta_{i1}, x_i, \beta_{*1}^{(0)}, \sigma_{11}^{(0)}). \\
  \end{align}
  Still we use the constraints in \eqref{eq:c0} to
  \eqref{eq:cs}. $\Delta_{i2}$ can be solved through the above
  equation, but with some difficulties:
  \begin{itemize}
  \item newton-Raphson method.
  \item numerical integration is needed for newton-raphson method to
    calculate the integral in above equation.
  \item numerical method to get the first derivative when trying to
    use newton-raphson method.
  \end{itemize}
\end{enumerate}

After we found ways to calculate $\Delta_{i1}$ and $\Delta_{i2}$, we
can use the \textbf{block Gibbs sampling} method to draw sample from
posterior distribution. Denote all the parameters (including
sensitivity parameters) to sample as : (omit the priors of them to
save typing so far)
\begin{displaymath}
  \bm \xi = \left( (\gamma_{01}, \gamma_{11}),
    (\gamma_{02}, \gamma_{12}), (\beta_{01}^{(1)}, \beta_{11}^{(1)}),
    (\beta_{02}^{(0)}, \beta_{12}^{(0)}, \beta_{22}^{(0)}),
    \sigma_{11}^{(1)}, \sigma_{11}^{(0)}, \sigma_{2|1}^{(1)}, \lambda,
    \phi \right).
\end{displaymath}
Bracketed parameters are marked to sample as a block in block Gibbs
sampling.  All the procedures require Metropolis-Hasting algorithm to
update samples since the likelihood is extremely complicated.

If MAR is assumed with no uncertainty, then $\beta_{02}^{(0)} =
\beta_{12}^{(0)} = \beta_{22}^{(0)} = 0$ and $\lambda = 1$. Details
for updating parameters:

\begin{itemize}
\item ($\bm \gamma_{01}, \bm \gamma_{11}$): using Metropolis-Hasting
  algorithm.
  \begin{enumerate}
  \item Draw ($\gamma_{01}^c, \gamma_{11}^c$) candidates from
    candidate distribution
  \item Based on the new candidate parameter $\bm \xi^c$, calculate
    candidate $\Delta_{i1}^c$ from equation (\ref{eq:solve1}) for each
    subject $i$ . If $R = 1$ for corresponding subject $i$, update
    candidate $\Delta_{i2}^c$ as well. (For $R = 0$, only need to
    update $\Delta_{i1}^c$).
  \item Plug in $\Delta_{i1}^c$ or ($\Delta_{i1}^c, \Delta_{i2}^c$) in
    likelihood (\ref{eq:like0}) or likelihood (\ref{eq:like1}) for $R
    = 0$ or $R = 1$ to get candidate likelihood.
  \item Obtain Metropolis-Hasting ratio and move the chain.
  \end{enumerate}
\item ($\gamma_{02}, \gamma_{12}$): similar algorithm but only update
  $\Delta_{i2}$ for subjects with $R = 1$. .
\item ($\beta_{01}^{(1)}, \beta_{11}^{(1)}$) : similar to
  ($\gamma_{01}, \gamma_{11}$).
\item For the rest of the parameters, algorithms for updating the
  samples are all similar to the first one.
\end{itemize}

\section{Trivariate normals under ignorability}
Let $\bm Y_i = (Y_{i1}, Y_{i2}, Y_{i3}) $ and follow-up time $S = \{1,
2, 3\}$. Covariates $x_i$ are constant and marginalized quantile
regression coefficients are assumed to be time varying $\bm \gamma =
(\bm \gamma_1^{\tau}, \bm \gamma_2^{\tau}, \bm \gamma_3^{\tau})$:
\begin{align*}
  \bm \gamma_1^{\tau} & = (\gamma_{01}^{\tau}, \gamma^{\tau}_{11}), \\
  \bm \gamma_{2}^{\tau} & = (\gamma_{02}^{\tau}, \gamma^{\tau}_{12}), \\
  \bm \gamma_{3}^{\tau} & = (\gamma_{03}^{\tau}, \gamma^{\tau}_{13}).
\end{align*}
Distribution of $Y_{i1}$ follows :
\begin{align}
  \label{eq:tri1}
  Y_{i1}| S = 1, x_i & \sim N(\Delta_{i1} + \beta_{01}^{(1)} + x_i\beta_{11}^{(1)}, \sigma^{(1)}), \\
  Y_{i1}| S = 2, x_i & \sim N(\Delta_{i1} + \beta_{01}^{(2)} + x_i\beta_{11}^{(2)}, \sigma^{(2)}), \\
  Y_{i1}| S = 3, x_i & \sim N(\Delta_{i1} + \beta_{01}^{(3)} + x_i\beta_{11}^{(3)}, \sigma^{(3)}). \\
\end{align}
Note, for identifiability issue, we require constraints:
\begin{align}
  \label{eq:constbi1}
  \beta_{01}^{(1)} + \beta_{01}^{(2)} + \beta_{01}^{(3)}  & = 0 ,\\
  \beta_{11}^{(1)} + \beta_{11}^{(2)} + \beta_{11}^{(3)} & = 0 .
\end{align}
Distribution of $Y_{i2} | Y_{i1}$:
\begin{align}
  \label{eq:tri2}
  * Y_{i2} | y_{i1} , S = 1, x_i & \sim N(\Delta_{i2} + \beta_{02}^{(1)} + x_i\beta_{12}^{(1)} + y_{i1}\beta_{22}^{(1)}, \tau_2^{(1)}),  \\
  Y_{i2} | y_{i1} , S \geq 2, x_i & \sim N(\Delta_{i2} + \beta_{02}^{(\geq 2)} +
  x_i\beta_{12}^{(\geq 2)} + y_{i1}\beta_{22}^{(\geq 2)}, \tau_2^{(\geq 2)}).
\end{align}
$*$ means the observation $Y_{i2}$ is not observed when the follow-up
time is 1. Same reason due to identifiability issue, constraints are:
\begin{align}
  \label{eq:constbi2}
  \beta_{02}^{(1)} + \beta_{02}^{(\geq 2)} & = 0 , \\
  \beta_{12}^{(1)} + \beta_{12}^{(\geq 2)} & = 0 , \\
  \beta_{22}^{(1)} + \beta_{22}^{(\geq 2)} & = 0 . \\
\end{align}
Distribution of $Y_{i3} | Y_{i2}, Y_{i1}$:
\begin{align}
  \label{eq:tri3}
  * Y_{i3} | y_{i2}, y_{i1} , S = 1, x_i & \sim N(\Delta_{i3} + \beta_{03}^{(1)} + x_i\beta_{13}^{(1)} + y_{i1}\beta_{23}^{(1)} + y_{i2}\beta_{33}^{(1)}, \tau_3^{(1)}),  \\
  * Y_{i3} | y_{i2}, y_{i1} , S = 2, x_i & \sim N(\Delta_{i3} + \beta_{03}^{(2)} + x_i\beta_{13}^{(2)} + y_{i1}\beta_{23}^{(2)} + y_{i2}\beta_{33}^{(2)}, \tau_3^{(2)}),  \\
  Y_{i3} | y_{i2}, y_{i1} , S = 3, x_i & \sim N(\Delta_{i3} + \beta_{03}^{(3)} +
  x_i\beta_{13}^{(3)} + y_{i1}\beta_{23}^{(3)} + y_{i2}\beta_{33}^{(3)}, \tau_3^{(3)}).
\end{align}
$*$ means the observation $Y_{i3}$ is not observed when the follow-up
time is 1 or 2. Constraints are:
\begin{align}
  \label{eq:constbi3}
  \beta_{03}^{(1)} + \beta_{03}^{(2)}  + \beta_{03}^{(3)}& = 0 , \\
  \beta_{13}^{(1)} + \beta_{13}^{(2)}  + \beta_{13}^{(3)}& = 0 , \\
  \beta_{23}^{(1)} + \beta_{23}^{(2)}  + \beta_{23}^{(3)}& = 0 , \\
  \beta_{33}^{(1)} + \beta_{33}^{(2)} + \beta_{33}^{(3)}& = 0 .
\end{align}
Sensitivity parameters are :
\begin{align*}
  \bm \xi_{s2} &= (\beta_{02}^{(1)}, \beta_{12}^{(1)}, \beta_{22}^{(1)}, \tau_2^{(1)}), \\
  \bm \xi_{s3} &= (\beta_{03}^{(1)},\beta_{03}^{(2)},  \beta_{13}^{(1,2)}, \beta_{23}^{(1, 2)}, \beta_{33}^{(1,2)}, \tau_2^{(1, 2)}) \\
\end{align*}
While $\bm \xi_{s2} = (0, 0, 0, \tau_2^{(\geq 2)})$ and $\bm \xi_{s3} = (0, 0, 0,
0, 0, 0, 0, 0, \tau_3^{(3)}, \tau_3^{(3)})$ yield MAR.
\section{Generalized multivariate quantile regression with monotone missingness }
% \subsection{glm}
% \subsection{pattern mixture}
% \subsection{$\exp(\lambda)$}

\section{TODO}

correlation within subject

\section{Computation}
\begin{itemize}
\item Denote
  \begin{displaymath}
    I(a,b) = \int \Phi \left( \frac{x-b}{a} \right)\phi(x) dx 
  \end{displaymath}
  where $\Phi$ is the standard normal cdf and $\phi$ is the pdf and $a > 0$.
  \begin{align*}
    \frac{\partial I(a,b)}{\partial b} & = - \frac{1}{a} \int \phi \left( \frac{x-b}{a} \right) \phi(x) dx \\
    & = - \frac{1}{\sqrt{2 \pi} \sqrt{a^2+1}} \exp \left( - \frac{b^2}{2(a^2+1)} \right)\\
    & = -\frac{1}{\sqrt{a^2+1}} \phi \left( \frac{b}{\sqrt{a^2+1}} \right)
  \end{align*}
  Since $I(a, \infty) = 0$,
  \begin{align}
    I(a,b) &= - \frac{1}{\sqrt{a^2+1}} \int_b^{\infty} \phi \left( \frac{s}{\sqrt{a^2+1}} \right) ds \nonumber \\
    &= \int_{b/\sqrt{a^2+1}}^{\infty} \phi(t) dt \nonumber\\
    \label{eq:int}
    & = 1- \Phi(b/\sqrt{a^2+1})
  \end{align}
  For $a < 0$,
  \begin{align*}
    \frac{\partial I(a,b)}{\partial b} & = - \frac{1}{a} \int \phi \left( \frac{x-b}{a} \right) \phi(x) dx \\
    & = - \frac{sgn(a)}{\sqrt{2 \pi} \sqrt{a^2+1}} \exp \left( - \frac{b^2}{2(a^2+1)} \right)\\
    & = -\frac{sgn(a)}{\sqrt{a^2+1}} \phi \left( \frac{b}{\sqrt{a^2+1}}
    \right)
  \end{align*}
  Since $I(a, -\infty) = 0$:
  \begin{align}
    I(a,b) &= \int^{b/\sqrt{a^2+1}}_{-\infty} \phi(t) dt \nonumber\\
    \label{eq:intneg}
    & = \Phi(b/\sqrt{a^2+1})
  \end{align}
\item If integrating over non-standard normal distribution:
  \begin{align*}
    \int \Phi(x)d\Phi(x; \mu, \sigma) & = \int \Phi(x) \frac{1}{\sigma} \phi \left( \frac{x-\mu}{\sigma} \right) dx \\
    & = \int \Phi(\sigma t + \mu)\phi(t) dt \\
    \text{by equation (\ref{eq:int}): } \quad & = 1 -
    \Phi(-\mu/\sigma/\sqrt{1/\sigma^2+1})
  \end{align*}
\item If integrating a non standard cdf over a non-standard normal
  distribution:
  \begin{align}
    \int \Phi \left( \frac{x-b}{a} \right) d\Phi(x; \mu, \sigma) & = \int \Phi \left( \frac{x-b}{a} \right) \frac{1}{\sigma} \phi \left( \frac{x-\mu}{\sigma} \right) dx \nonumber\\
    &= \int \Phi \left( \frac{\sigma y + \mu - b}{a}  \right) \phi(y) dy \nonumber \\
    \label{eq:intg1}
    & = 1- \Phi \left( \frac{b-\mu}{\sigma} / \sqrt{\frac{a^2}{\sigma^2}+1} \right)
  \end{align}
  If $a < 0$,
  \begin{equation}
    \label{eq:intg2}
    \int \Phi \left( \frac{x-b}{a} \right) d\Phi(x; \mu, \sigma) = \Phi \left( \frac{b-\mu}{\sigma} / \sqrt{\frac{a^2}{\sigma^2}+1} \right)
  \end{equation}

\end{itemize}

For computation of higher order $\Delta_{ij}$ for $j >= 3$, we propose two
approaches: (for convenience, we use $\Delta_j$ to save typing for
subject $i$)

\begin{enumerate}
\item \textbf{Assume first order relationship:} We assume
  \begin{equation}
    \pr (Y_j|S, x, Y_{j-1}, \ldots, Y_1) = \pr (Y_j|S, x, Y_{j-1})
  \end{equation}
  After obtaining $\Delta_{j-1}$
  \begin{align*}
    \pr (Y_j \leq  x\gamma | S = s) & =  \int\dots\int \pr (Y_j \leq x\gamma | S=s, x, Y_{j-1}, \ldots, Y_1) dF(Y_{j-1}|S=s, Y_{j-2}, \ldots, Y_1)\\
    & \quad  \cdots dF(Y_2|S=s, Y_1) \\
    & = \int \pr (Y_j \leq x\gamma | S=s, x, Y_{j-1}) dF(Y_{j-1}|S=s, Y_{j-2})
  \end{align*}

  Thus, only one integral is needed. Furthermore, by previous
  computation for integral of a normal cdf over normal pdf, we can
  simplify above by 'close' form.

\item \textbf{Recursive Computation: }

  From equation \eqref{eq:intg1} or \eqref{eq:intg2}, we can find
  after integration, the kernel part is still a normal cdf, but with
  other coefficients. So recursive simplification can be applied. Here
  we take trivariate case as an example:

  For convenience, we only calculate $\pr (Y_3 \leq x\gamma | S= 1)$, omit
  superscript and denote $\beta_{ij}^{(1)} $ as $\beta_{ij}$ to save
  typing.  The model we have is
  \begin{align*}
    Y_3 | S= 1 , Y_2, Y_1 & \sim N(\Delta_3 + \mu_3 + y_2\beta_{23} + y_1\beta_{13}, \tau_3) \\
    Y_2 | S = 1, Y_{1}  & \sim N(\Delta_2 + \mu_2 + y_1\beta_{12}, \tau_2)                \\
    Y_1 | S= 1 & \sim N(\Delta_1 + \mu_1, \sigma)
  \end{align*}
  Assume $\beta_{23}, \beta_{13}, \beta_{12} > 0 $, then
  \begin{align*}
    \pr (Y_3 \leq x\gamma |S = 1) & = \int \int \pr (Y_3 \leq x\gamma|S = 1, y_2, y_1) f(y_2|S=1, y_1) f(y_1|S=1) dy_2dy_1\\
    & = \int \left[ \int \Phi \left( \frac{x\gamma - \Delta_3 - \mu_3 - y_2\beta_{23}- y_1\beta_{13}}{\tau_3} \right) f(y_2; \Delta_2+\mu_2+y_1\beta_{12}, \tau_2)dy_2 \right]\\
    & \quad f(y_1|S=1) dy_1\\
    & = \int \left[ \int \Phi \left( \frac{y_2 - (x\gamma - \Delta_3 - \mu_3 - y_1\beta_{13})/\beta_{23}}{-\tau_3/\beta_{23}} \right) f(y_2; \Delta_2+\mu_2+y_1\beta_{12}, \tau_2)dy_2 \right] \\
    & \quad f(y_1|S=1) dy_1\\
    \text{by \eqref{eq:intg2} } \quad & = \int \Phi \left( \frac{(x\gamma - \Delta_3 - \mu_3 - y_1\beta_{13})/\beta_{23} - \Delta_2 - \mu_2 - y_1\beta_{12}}{\tau_2 \sqrt{\tau_3^2/\beta_{23}^2/\tau_2^2 + 1}} \right)f(y_1; \Delta_1+ \mu_1, \sigma) dy_1\\
    & = \int \Phi \left( \frac{y_1 - b^{*}}{a^{*}} \right) f(y_1; \mu^{*}, \sigma)
    dy_1
  \end{align*}
  where again the last equation has closed form from equation
  \eqref{eq:intg1} or equation \eqref{eq:intg2}.
\end{enumerate}

\end{document}
