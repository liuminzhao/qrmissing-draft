
\documentclass[12pt]{article}
\usepackage{amsmath}
\usepackage{geometry}
% \usepackage{times}
\usepackage{graphicx}
% \usepackage{courier}
\usepackage{mathpazo}
\usepackage{mathrsfs}
\usepackage{bm}
\usepackage[colorlinks,linkcolor=red,anchorcolor=blue,citecolor=blue]{hyperref}
\usepackage{amsthm}
\geometry{verbose,letterpaper,tmargin=1in,bmargin=.75in,lmargin=.75in,rmargin=1in}

\title{Simulation (t error)}
\date{\today}
\author{}

\newcommand{\polya}{P'{o}lya}
\newcommand{\iid}{\stackrel{\text{i.i.d}}{\sim}}
\DeclareMathOperator{\pr}{p}
\DeclareMathOperator{\pt}{PT}

\begin{document}

\section{T error}
We want to simulate t errors.

\textbf{Remark 1:}
If $X|W \sim N(0, W)$, $W \sim IG(\nu/2, \nu/2)$, then $X \sim t_{\nu}$.

Now we introduce a latent variable $W$, such that
\begin{align}
Y_1 | R = 1, W, x & \sim  N(\beta_{10}^{(1)} + \beta_{11}^{(1)}x, W) \\
Y_2 | R = 1, W, x, Y_1 &\sim  N(\beta_{20}^{(1)} + \beta_{21}^{(1)}x + \beta_{22}^{(1)}Y_1, 3W/4)
\end{align}
where $R$ stands for the missingness indicator.

Integrate $Y_1|R = 1, W, x$ out of (2):
\begin{align}
Y_2 | R = 1, W, x \sim N ( (\beta_{20}^{(1)} + \beta_{22}^{(1)}\beta_{10}^{(1)}) + (\beta_{21}^{(1)} + \beta_{22}^{(1)}\beta_{11}^{(1)})x, ((\beta_{22}^{(1)}))^2 + 3/4)W)
\end{align}
Let $\beta_{22}^{(1)} = 1/2$, thus
\begin{align}
Y_2 | R = 1, W, x \sim N ( (\beta_{20}^{(1)} + \beta_{22}^{(1)}\beta_{10}^{(1)}) + (\beta_{21}^{(1)} + \beta_{22}^{(1)}\beta_{11}^{(1)})x, W)
\end{align}
Now use remark 1, we have
\begin{align}
Y_1|R = 1, x & \sim \beta_{10}^{(1)} + \beta_{11}^{(1)}x + \epsilon_1 \\
Y_2|R = 1, x & \sim (\beta_{20}^{(1)} + \beta_{22}^{(1)}\beta_{10}^{(1)}) + (\beta_{21}^{(1)} + \beta_{22}^{(1)}\beta_{11}^{(1)})x + \epsilon_2
\end{align}
where $\epsilon_1, \epsilon_2 \sim t_{\nu}$

By similar approach, we can define distribution for $R = 0$:
\begin{align}
Y_1|R = 0, x , W& \sim \beta_{10}^{(0)} + \beta_{11}^{(0)}x + N(0, W)
\end{align}
and if we further assume $p(Y_2 | R= 1, W, x) = p(Y_2|R = 0, W, x) $, we
will have $p(Y_2|R = 1, x) = p(Y_2|R =0, x)$, which yields MAR.

Therefore, the complete sampling plan for t error with MAR can be :
\begin{enumerate}
\item $W \sim IG(\nu/2, \nu/2)$
\item $R \sim Bernoulli(\pi)$
\item draw $Y_1$ based on R and  (1) or (7)
\item draw $Y_2$ based on (2) , regardless R due to MAR
\end{enumerate}

As we mentioned in 0429.pdf, we have the linear form for marginal  quantile lines for the mixture of t distributions.

\newpage

\section{ALD error}
Kuzobowski and Podgorski (2000) point out if
\begin{align*}
\xi & \sim Gamma(1, \tau) (\text{rate}) \\
\epsilon_p | \xi & \sim N \left( \frac{1 - 2p}{p(1 - p)}\xi, \frac{2\xi}{\tau p (1- p)} \right)
\end{align*}
then marginally $\epsilon_p \sim ALD(p, 0, \tau)$, where $\tau$ is a scale parameter and $Pr(\epsilon_p < 0 ) = p$.

Therefore, we still introduce a latent variable $\xi$. Let
\begin{align}
Y_1 | R = 1, \xi & \sim x\beta_1^{(1)} + N\left( \frac{1 - 2p}{p(1 - p)}\xi, \frac{2\xi}{\tau p (1- p)} \right) \\
Y_2 | R = 1, \xi, Y_1 & \sim x \beta_2^{(1)} + \frac{1}{2} Y_1 +  N\left( \frac{1}{2}\frac{1 - 2p}{p(1 - p)}\xi, \frac{2\xi}{\tau p (1- p)}* \frac{3}{4} \right) \label{eq:2}
\end{align}
In (\ref{eq:2}), integrate $Y_1|R = 1, \xi$ out:
\begin{displaymath}
Y_2 | R = 1, \xi    \sim x(\beta_2^{(1)} + \frac{1}{2}\beta_1^{(1)}) +  N\left(\frac{1 - 2p}{p(1 - p)}\xi, \frac{2\xi}{\tau p (1- p)} \right)
\end{displaymath}
Thus by Kuzobowski and Podgorski (2000),
\begin{align*}
Y_1 | R= 1 & \sim ALD(p, x\beta_{1}^{(1)}, \tau) \\
Y_2 | R = 1 & \sim  ALD(p, x(\beta_2^{(1)} + \frac{1}{2}\beta_1^{(1)}) , \tau)
\end{align*}
Using the similar techniques in previous section, we propose a sufficient condition to yield MAR. Let $p(Y_2 | R = 1, \xi, Y_1 ) = p(Y_2|R = 0, \xi, Y_1$, then $p(Y_2|R = 1, Y_1) = p(Y_2|R = 0, Y_1)$.

Propose distribution for $Y_1|R = 0$:
\begin{equation}
\label{eq:3}
Y_1|R = 0, \xi \sim x\beta_1^{(0)} +  N\left( \frac{1 - 2p}{p(1 - p)}\xi, \frac{2\xi}{\tau p (1- p)} \right)
\end{equation}
Then the complete sampling procedure for Asymmetric laplace distribution with MAR could be:
\begin{enumerate}
\item draw $\xi \sim Gamma(1, \tau)$
\item draw $R \sim Bernoulli(\pi)$
\item draw $Y_1 |R, \xi$ from (8) or (\ref{eq:3})
\item draw $Y_2 | Y_1, R, \xi$ from (\ref{eq:2})
\end{enumerate}
Thus within each pattern (R) , $Y_1$ and $Y_2$ are ALD, and by method proposed in 0429.pdf, we can get the linear form for marginal quantile regression line for mixture of ALD distribution.

\end{document}
%%% Local Variables:
%%% mode: latex
%%% TeX-master: t
%%% End:
